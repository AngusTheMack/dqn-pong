% 
\documentclass[10pt]{article}
\usepackage{amscd,amsfonts,amssymb,amstext,latexsym} 
\usepackage{amsmath,mathbbol,mathrsfs,stmaryrd, mathtools} 
%\usepackage{mathbbol,mathrsfs,stmaryrd}
\usepackage {algorithm2e} 
\usepackage{theoremref}
\usepackage[T1]{fontenc}
\usepackage[english]{babel} 
\usepackage {enumerate}
\usepackage{url}
%\usepackage {algpseudocode}  
\usepackage{graphics} 
\usepackage{tikz}
%\usepackage[square]{natbib}
\usepackage[width=14.8cm,left=3cm]{geometry}
\usetikzlibrary{automata,calc}
%\usepackage{tgtermes} 
\usepackage{listings}
\usepackage{mathptmx}
\usepackage{fancyhdr}
\usepackage{verbatim}
\usepackage{enumitem}
\usepackage{booktabs}
\usepackage[flushleft]{threeparttable}
\usepackage{listings}
\usepackage{verbatim}
\usepackage{fancyhdr}
\usepackage{multirow,multicol}
\usepackage[colorlinks=true,linkcolor=blue,citecolor=blue,urlcolor=blue]{hyperref}
\usepackage{tabto}
\lstset{ %
language=C++,                % choose the language of the code
basicstyle={\ttfamily},       % the size of the fonts that are used for the code
backgroundcolor=\color{white},  % choose the background color. You must add \usepackage{color}
showspaces=false,               % show spaces adding particular underscores
aboveskip=6mm, 
%belowskip=3mm, 
numbers=left, numberfirstline=false, numberblanklines=false,
numberstyle=\tiny\color{gray}, numbersep= 5pt, 
showstringspaces=false,         % underline spaces within strings
showtabs=false,                 % show tabs within strings adding particular underscores
%frame=single,           % adds a frame around the code
%frame = tb, 
frame = none, 
tabsize=2,          % sets default tabsize to 2 spaces
captionpos=b,           % sets the caption-position to bottom
breaklines=true,        % sets automatic line breaking
breakatwhitespace=false,    % sets if automatic breaks should only happen at whitespace
escapeinside={\%*}{*)}          % if you want to add a comment within your code
}
%\graphicspath{{../../pics/}}
\fancypagestyle{plain}{
\fancyhf{}
\rhead{School of Computer Science and Applied Mathematics\\ 
%\noindent\rule{15.4cm}{0.4pt}\\
\footnotesize{\textsc{University of the Witwatersrand, Johannesburg}}}
\lhead{\includegraphics[scale=0.08]{witslogo_h.png}}
\fancyfoot[C]{\thepage}
\renewcommand{\headrulewidth}{0.4pt}
}

\textwidth=16.8cm 
\textheight=22.6cm 
\evensidemargin 0pt 
\oddsidemargin 0pt 
\leftmargin 0pt 
\rightmargin 0pt 
\setlength{\topmargin}{0pt} 
\setlength{\footskip}{50pt}
\setlength{\parindent}{0pt}
\setlength{\parskip}{1em}
\linespread{1} 
% 
\makeatletter
\newcommand{\rmnum}[1]{\romannumeral #1}
\newcommand{\Rmnum}[1]{\expandafter\@slowromancap\romannumeral #1@}
\makeatother

\begin{document}
\title{COMS4047A DQN Project}
\author{Angus Mackenzie - 1106817}
\date{\today} 
\maketitle 
%\thispagestyle{empty}
\pagestyle{fancy}
\fancyhf{}
\fancyhead[R]{\thepage}
\fancyhead[L]{COMS4040A}
%\vskip 3mm 
%\pagenumbering{roman}
%\newpage
\pagenumbering{arabic}
\section{Introduction}
This lab explores the implementation of a Deep Q-Learning (DQN) agent to play pong through the use of OpenAI \cite{openai}. Initially we were provided source code, which acted as the base, and then a solution was built on top to the specification of \cite{Deepmind} and \cite{NaturePaper}.

\section{Method}
A model was created in order to train the DQN, this model used the same architecture as that of \cite{NaturePaper}. This network takes in an input of 4 frames of 84x84 pixels each as the state representation, and then has an output for each possible action. The first layer of the network has a stride of 4, and an input of 16 8x8 filters. The second layer has an input of 32 with a stride of 2 and 4x4 filters. The Third layer has an input of 64, and a filter of 3x3 with a stride of 1. The output is simply the fully connected linear layer that has an output for each action. 

The model was then trained for 1 million steps, and the rewards for every 100 steps tracked. The model was also saved every 100 steps. 

\section{Results}

\includegraphics{../submission/plot.png}H

As shown above, the model started learning slowly, with initial rewards per episode beginning around -20. After training for roughly 2 hours the model increases the reward per episode to roughly 20.


%\bibliographystyle{apalike} I prefer plain references, but I will ask Hairong if that is okay
\bibliographystyle{plain}
\bibliography{references}
\end{document} 
